\documentclass[12pt,a4paper]{article}

% Paquetes básicos
\usepackage[utf8]{inputenc}
\usepackage[spanish]{babel}
\usepackage{amsmath,amsfonts,amssymb}
\usepackage{graphicx}
\usepackage{hyperref}
\usepackage{booktabs}

% Configuración de márgenes
\usepackage[margin=2.5cm]{geometry}

% ── Comandos reutilizables (Principio de Abstracción) ────────
% Defínelos una vez aquí, úsalos en todo el documento.
% Si un valor cambia, solo lo cambias en un lugar.
\newcommand{\numparticipantes}{4}
\newcommand{\nombreproyecto}{Proyecto de Ejemplo}

% Título y autores
\title{\nombreproyecto: Análisis de Datos}
\author{
    Investigador Principal\textsuperscript{1} \and
    Colaborador\textsuperscript{1}
}
\date{\today}

\begin{document}

\maketitle

\begin{abstract}
Este documento es una plantilla que demuestra los principios de
\textit{Code and Data for the Social Sciences} (Gentzkow \& Shapiro, 2014)
aplicados a un artículo académico en \LaTeX. La estructura del proyecto
separa código, datos y documentos. Todos los resultados incluidos se
generan automáticamente desde los scripts de análisis.
\end{abstract}

\section{Introducción}

Este artículo utiliza una muestra de \numparticipantes{} participantes.
Los datos fueron procesados mediante scripts automatizados siguiendo los
principios de reproducibilidad.

\section{Datos y Métodos}

Los datos crudos se encuentran en \texttt{datos/crudos/} y nunca se
modifican directamente. La limpieza se realiza con el script
\texttt{codigo/01\_limpiar.py}, y el análisis con
\texttt{codigo/02\_analizar.py}.

Todo el pipeline se ejecuta con un solo comando:

\begin{verbatim}
make
\end{verbatim}

\section{Resultados}

% La siguiente tabla se genera automáticamente por 02_analizar.py
% Si los datos cambian, solo hay que ejecutar "make" de nuevo.
\input{../resultados/tablas/resumen}

\section{Conclusión}

Esta plantilla demuestra que es posible mantener un proyecto de
investigación organizado, reproducible y colaborativo usando las
herramientas del taller (Git, GitHub, VS Code) junto con los principios
de Gentzkow y Shapiro.

\end{document}
