% ============================================================================
% main.tex — Archivo principal del artículo
%
% Principio G&S Capítulo 2 (Automatización):
%   Este archivo se compila con un solo comando: make (Linux/Mac) o
%   compilar.bat (Windows). No se necesita intervención manual.
%
% Principio G&S Capítulo 4 (Directorios):
%   Las secciones viven en sections/, las figuras en figures/.
%   Todas las referencias son locales (portables).
%
% Principio G&S Capítulo 7 (Documentación):
%   El código es auto-documentado. Los comentarios explican POR QUÉ,
%   no QUÉ. Los nombres de comandos declaran su función.
% ============================================================================
\documentclass[12pt,a4paper]{article}

% --- Codificación e idioma ---
\usepackage[utf8]{inputenc}
\usepackage[spanish]{babel}

% --- Matemáticas ---
\usepackage{amsmath,amsfonts,amssymb}

% --- Figuras (rutas portables — G&S Cap. 4) ---
\usepackage{graphicx}
\graphicspath{{figures/}}

% --- Referencias cruzadas e hipervínculos ---
\usepackage{hyperref}
\usepackage{natbib}

% --- Tablas ---
\usepackage{booktabs}

% --- Márgenes ---
\usepackage[margin=2.5cm]{geometry}

% ============================================================================
% COMANDOS REUTILIZABLES — G&S Capítulo 6 (Abstracción)
%
% Estos comandos eliminan redundancia: si cambia el formato de cómo
% presentamos una herramienta o un comando, se cambia en UN solo lugar.
% ============================================================================

% Herramientas del flujo de trabajo
\newcommand{\herramienta}[1]{\textbf{#1}}
% Comandos de terminal
\newcommand{\comando}[1]{\texttt{#1}}
% Nombres de archivo
\newcommand{\archivo}[1]{\texttt{#1}}
% Principios de Gentzkow & Shapiro
\newcommand{\principio}[1]{\textsc{#1}}
% Referencia corta al artículo
\newcommand{\gs}{Gentzkow y Shapiro (2014)}

% ============================================================================
% METADATOS
% ============================================================================
\title{Colaboración Académica con Git y \LaTeX:\\
       Un Flujo de Trabajo Basado en Buenas Prácticas}
\author{
    Mauricio\textsuperscript{1} \and
    José Miguel\textsuperscript{1} \and
    Rodrigo\textsuperscript{1}
}
\date{\today}

% ============================================================================
% DOCUMENTO
% ============================================================================
\begin{document}

\maketitle

\begin{abstract}
Este artículo presenta un flujo de trabajo colaborativo para la escritura
de artículos académicos en \LaTeX{}, integrando principios de ingeniería
de software propuestos por \gs{}. El flujo combina \herramienta{Git} para
control de versiones, \herramienta{GitHub} para colaboración,
\herramienta{VS Code} para edición local y \herramienta{Overleaf} para
verificación en la nube. Se incluyen scripts de automatización que
permiten compilar el documento completo con un solo comando.
\end{abstract}

% Secciones modulares — G&S Cap. 4: separar por función
\section{Introducción}

% === JOSÉ MIGUEL: Edita esta sección ===

Este es el texto inicial de la introducción. 

José Miguel reemplazará este contenido con una introducción sobre la importancia de la colaboración en proyectos académicos.

\subsection{Motivación}

[Pendiente: explicar por qué es importante tener un flujo de trabajo colaborativo]

\subsection{Objetivos}

[Pendiente: listar los objetivos del artículo]

\section{Trabajos Relacionados}
\label{sec:trabajos-relacionados}

Existen diversas aproximaciones a la colaboración en documentos académicos.
A continuación revisamos las más relevantes.

\subsection{Herramientas de edición colaborativa}
\label{sec:herramientas-colaborativas}

Google Docs y Microsoft Word Online permiten edición simultánea, pero carecen
de soporte nativo para \LaTeX{}. \herramienta{Overleaf} \citep{overleaf2024docs}
resuelve esto parcialmente al ofrecer un editor \LaTeX{} en línea con
colaboración en tiempo real.

\subsection{Control de versiones en academia}
\label{sec:control-versiones-academia}

El uso de \herramienta{Git} en investigación ha crecido significativamente.
\citet{perez2024github} demuestran que \herramienta{GitHub} facilita la
reproducibilidad y colaboración en laboratorios de investigación.

\subsection{Buenas prácticas de ingeniería de software}
\label{sec:buenas-practicas}

\citet{gentzkow2014code} proponen un conjunto de principios para organizar
código y datos en ciencias sociales: \principio{Automatización},
\principio{Control de versiones}, \principio{Directorios},
\principio{Claves}, \principio{Abstracción}, \principio{Documentación}
y \principio{Gestión}. Nuestro flujo de trabajo integra estos principios
en el contexto de la escritura colaborativa en \LaTeX{}.

\subsection{Flujos de trabajo híbridos}
\label{sec:flujos-hibridos}

Algunos equipos combinan múltiples herramientas. El enfoque que presentamos
en este artículo sigue esta línea, integrando \herramienta{Overleaf},
\herramienta{GitHub} y editores locales como \herramienta{VS Code}.

\section{Metodología}

En esta sección describimos las herramientas utilizadas y el flujo de trabajo 
propuesto para la colaboración en artículos académicos con LaTeX.

\subsection{Herramientas utilizadas}

El flujo de trabajo propuesto se basa en cuatro herramientas principales:

\begin{itemize}
    \item \textbf{Git}: Sistema de control de versiones distribuido que permite 
    rastrear cambios en archivos de texto, crear ramas para desarrollo paralelo, 
    y fusionar contribuciones de múltiples autores.
    
    \item \textbf{GitHub}: Plataforma en la nube que aloja repositorios Git, 
    proporcionando respaldo automático, Pull Requests para revisión de cambios, 
    e Issues para gestión de tareas.
    
    \item \textbf{VS Code}: Editor de código con extensión LaTeX Workshop que 
    permite compilación local automática y vista previa del PDF en tiempo real.
    
    \item \textbf{Overleaf}: Editor LaTeX en línea con compilador en la nube, 
    utilizado como capa de verificación final mediante sincronización con GitHub.
\end{itemize}

\subsection{Flujo de trabajo}

El ciclo diario de trabajo sigue el orden: \textbf{Local $\rightarrow$ GitHub 
$\rightarrow$ Overleaf}.

\begin{enumerate}
    \item \textbf{Obtener cambios}: \texttt{git pull origin main}
    \item \textbf{Editar localmente}: Escribir y compilar en VS Code
    \item \textbf{Preparar cambios}: \texttt{git add archivo.tex}
    \item \textbf{Guardar snapshot}: \texttt{git commit -m "descripción"}
    \item \textbf{Subir a GitHub}: \texttt{git push origin main}
    \item \textbf{Verificación final}: Sincronizar Overleaf con GitHub
\end{enumerate}

\section{Resultados}
\label{sec:resultados}

Implementamos el flujo de trabajo propuesto durante el desarrollo de este
artículo. A continuación presentamos los resultados observados.

\subsection{Beneficios de la colaboración con Git}
\label{sec:beneficios-git}

Durante las dos horas del taller, el equipo logró:
\begin{itemize}
    \item Configurar un entorno de trabajo compartido
    \item Realizar múltiples contribuciones en paralelo usando ramas
    \item Resolver conflictos de manera sistemática
    \item Mantener un historial completo de cambios
\end{itemize}

\subsection{Mapeo de principios G\&S al flujo de trabajo}
\label{sec:mapeo-gs}

La Tabla~\ref{tab:mapeo-gs} muestra cómo cada principio de \gs{} se
implementa en el flujo de trabajo propuesto.

\begin{table}[h]
\centering
\caption{Mapeo de principios de \gs{} al flujo de trabajo}
\label{tab:mapeo-gs}
\begin{tabular}{@{}lll@{}}
\toprule
\textbf{Principio G\&S} & \textbf{Implementación} & \textbf{Herramienta} \\
\midrule
Automatización (Cap.~2) & \comando{make} compila todo & Makefile \\
Control de versiones (Cap.~3) & \comando{make check} antes de push & Git \\
Directorios (Cap.~4) & Separación entrada/salida & .gitignore \\
Claves (Cap.~5) & Convención \comando{sec:}, \comando{fig:}, \comando{tab:} & \LaTeX{} \\
Abstracción (Cap.~6) & Comandos reutilizables & \archivo{main.tex} \\
Documentación (Cap.~7) & Código auto-documentado & Todos \\
Gestión (Cap.~8) & GitHub Issues + commits & GitHub \\
\bottomrule
\end{tabular}
\end{table}

\subsection{Comparación con métodos tradicionales}
\label{sec:comparacion}

La Tabla~\ref{tab:comparacion} compara el flujo tradicional con el propuesto.

\begin{table}[h]
\centering
\caption{Comparación de flujos de trabajo colaborativo}
\label{tab:comparacion}
\begin{tabular}{@{}lcc@{}}
\toprule
\textbf{Característica} & \textbf{Tradicional} & \textbf{Git + Overleaf} \\
\midrule
Control de versiones & Manual & Automático \\
Historial de cambios & Limitado & Completo \\
Trabajo simultáneo & Difícil & Fácil \\
Resolución de conflictos & Ad-hoc & Sistemática \\
Backup & Manual & Automático \\
Revisión de cambios & Por correo & Pull Requests \\
Compilación & Manual (4 pasos) & \comando{make} (1 paso) \\
\bottomrule
\end{tabular}
\end{table}

\section{Conclusión}
\label{sec:conclusion}

En este artículo presentamos un flujo de trabajo colaborativo para la
escritura de artículos académicos en \LaTeX{}, combinando \herramienta{Overleaf},
\herramienta{GitHub} y \herramienta{VS Code}, e integrando los principios
de buenas prácticas de \gs{}.

\subsection{Contribuciones principales}
\label{sec:contribuciones}

Las principales contribuciones de este trabajo son:
\begin{enumerate}
    \item Un flujo de trabajo que integra los siete principios de \gs{}
    \item Scripts de \principio{Automatización} que permiten compilar con
          un solo comando (\comando{make})
    \item Verificación pre-commit con \comando{make check}
          (\principio{Control de versiones})
    \item Estructura de directorios que separa entradas de salidas
          (\principio{Directorios})
    \item Comandos \LaTeX{} reutilizables que eliminan redundancia
          (\principio{Abstracción})
\end{enumerate}

\subsection{Limitaciones}
\label{sec:limitaciones}

El flujo propuesto requiere que al menos un miembro del equipo tenga
cuenta Premium de \herramienta{Overleaf} para la sincronización con
\herramienta{GitHub}. Además, existe una curva de aprendizaje inicial
para usuarios no familiarizados con \herramienta{Git}.

\subsection{Trabajo futuro}
\label{sec:trabajo-futuro}

Como trabajo futuro, se podría explorar:
\begin{itemize}
    \item Integración con sistemas de gestión de referencias como Zotero
    \item Automatización de compilación con GitHub Actions
    \item Plantillas pre-configuradas para diferentes journals
    \item Pre-commit hooks que ejecuten \comando{make check} automáticamente
\end{itemize}

\subsection*{Agradecimientos}

Agradecemos a todos los participantes del taller por su entusiasmo
y colaboración durante el desarrollo de este ejercicio práctico.


% Bibliografía
\bibliographystyle{apalike}
\bibliography{references}

\end{document}
