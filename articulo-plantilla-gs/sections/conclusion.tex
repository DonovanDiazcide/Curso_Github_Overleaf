\section{Conclusión}
\label{sec:conclusion}

En este artículo presentamos un flujo de trabajo colaborativo para la
escritura de artículos académicos en \LaTeX{}, combinando \herramienta{Overleaf},
\herramienta{GitHub} y \herramienta{VS Code}, e integrando los principios
de buenas prácticas de \gs{}.

\subsection{Contribuciones principales}
\label{sec:contribuciones}

Las principales contribuciones de este trabajo son:
\begin{enumerate}
    \item Un flujo de trabajo que integra los siete principios de \gs{}
    \item Scripts de \principio{Automatización} que permiten compilar con
          un solo comando (\comando{make})
    \item Verificación pre-commit con \comando{make check}
          (\principio{Control de versiones})
    \item Estructura de directorios que separa entradas de salidas
          (\principio{Directorios})
    \item Comandos \LaTeX{} reutilizables que eliminan redundancia
          (\principio{Abstracción})
\end{enumerate}

\subsection{Limitaciones}
\label{sec:limitaciones}

El flujo propuesto requiere que al menos un miembro del equipo tenga
cuenta Premium de \herramienta{Overleaf} para la sincronización con
\herramienta{GitHub}. Además, existe una curva de aprendizaje inicial
para usuarios no familiarizados con \herramienta{Git}.

\subsection{Trabajo futuro}
\label{sec:trabajo-futuro}

Como trabajo futuro, se podría explorar:
\begin{itemize}
    \item Integración con sistemas de gestión de referencias como Zotero
    \item Automatización de compilación con GitHub Actions
    \item Plantillas pre-configuradas para diferentes journals
    \item Pre-commit hooks que ejecuten \comando{make check} automáticamente
\end{itemize}

\subsection*{Agradecimientos}

Agradecemos a todos los participantes del taller por su entusiasmo
y colaboración durante el desarrollo de este ejercicio práctico.
