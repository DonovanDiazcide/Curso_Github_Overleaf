\section{Metodología}
\label{sec:metodologia}

En esta sección describimos las herramientas utilizadas y el flujo de trabajo
propuesto, organizado según los principios de \gs{}.

\subsection{Herramientas utilizadas}
\label{sec:herramientas}

El flujo de trabajo se basa en cuatro herramientas principales:

\begin{itemize}
    \item \herramienta{Git}: Sistema de control de versiones distribuido que
    permite rastrear cambios, crear ramas para desarrollo paralelo, y fusionar
    contribuciones de múltiples autores.

    \item \herramienta{GitHub}: Plataforma en la nube que aloja repositorios
    \herramienta{Git}, proporcionando respaldo automático, Pull Requests para
    revisión de cambios, e Issues para gestión de tareas
    (\principio{Gestión}, Cap.~8 de \gs{}).

    \item \herramienta{VS Code}: Editor de código con extensión LaTeX Workshop
    que permite compilación local automática y vista previa del PDF.

    \item \herramienta{Overleaf}: Editor \LaTeX{} en línea con compilador en
    la nube, utilizado como capa de verificación final mediante sincronización
    con \herramienta{GitHub}.
\end{itemize}

\subsection{Flujo de trabajo}
\label{sec:flujo-trabajo}

El ciclo diario sigue el orden: Local $\rightarrow$ GitHub $\rightarrow$
Overleaf. Antes de cada \comando{git push}, se ejecuta \comando{make check}
para verificar que el documento compila correctamente
(\principio{Control de versiones}, Cap.~3 de \gs{}).

\begin{enumerate}
    \item Obtener cambios: \comando{git pull origin main}
    \item Editar localmente en \herramienta{VS Code}
    \item Compilar con \comando{make} (\principio{Automatización})
    \item Verificar: \comando{make check}
    \item Preparar cambios: \comando{git add archivo.tex}
    \item Guardar snapshot: \comando{git commit -m "descripción"}
    \item Subir a \herramienta{GitHub}: \comando{git push origin main}
    \item Verificación final: sincronizar \herramienta{Overleaf}
\end{enumerate}

\subsection{Estructura de directorios}
\label{sec:estructura-directorios}

Siguiendo el principio de \principio{Directorios} (Cap.~4 de \gs{}),
los archivos se separan en entradas y salidas:

\begin{itemize}
    \item \archivo{sections/} --- contenido del artículo (entrada)
    \item \archivo{figures/} --- imágenes (entrada)
    \item \archivo{scripts/} --- automatización (entrada)
    \item \archivo{output/} --- PDF final (salida, no en Git)
    \item \archivo{temp/} --- archivos auxiliares (temporal, no en Git)
\end{itemize}
