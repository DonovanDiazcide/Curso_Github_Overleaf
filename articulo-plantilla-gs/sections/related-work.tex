\section{Trabajos Relacionados}
\label{sec:trabajos-relacionados}

Existen diversas aproximaciones a la colaboración en documentos académicos.
A continuación revisamos las más relevantes.

\subsection{Herramientas de edición colaborativa}
\label{sec:herramientas-colaborativas}

Google Docs y Microsoft Word Online permiten edición simultánea, pero carecen
de soporte nativo para \LaTeX{}. \herramienta{Overleaf} \citep{overleaf2024docs}
resuelve esto parcialmente al ofrecer un editor \LaTeX{} en línea con
colaboración en tiempo real.

\subsection{Control de versiones en academia}
\label{sec:control-versiones-academia}

El uso de \herramienta{Git} en investigación ha crecido significativamente.
\citet{perez2024github} demuestran que \herramienta{GitHub} facilita la
reproducibilidad y colaboración en laboratorios de investigación.

\subsection{Buenas prácticas de ingeniería de software}
\label{sec:buenas-practicas}

\citet{gentzkow2014code} proponen un conjunto de principios para organizar
código y datos en ciencias sociales: \principio{Automatización},
\principio{Control de versiones}, \principio{Directorios},
\principio{Claves}, \principio{Abstracción}, \principio{Documentación}
y \principio{Gestión}. Nuestro flujo de trabajo integra estos principios
en el contexto de la escritura colaborativa en \LaTeX{}.

\subsection{Flujos de trabajo híbridos}
\label{sec:flujos-hibridos}

Algunos equipos combinan múltiples herramientas. El enfoque que presentamos
en este artículo sigue esta línea, integrando \herramienta{Overleaf},
\herramienta{GitHub} y editores locales como \herramienta{VS Code}.
