\section{Introducción}
\label{sec:introduccion}

La colaboración efectiva es fundamental en la investigación académica moderna.
Los proyectos de investigación involucran cada vez más a equipos distribuidos
geográficamente, lo que hace necesario contar con herramientas que faciliten
el trabajo conjunto.

En particular, la escritura de artículos académicos presenta desafíos únicos:
múltiples autores necesitan editar el mismo documento, mantener un historial
de cambios, y asegurar que todos trabajen sobre la versión más reciente.

\subsection{Motivación}
\label{sec:motivacion}

El flujo de trabajo tradicional basado en enviar archivos por correo electrónico
presenta numerosos problemas: versiones duplicadas, pérdida de cambios, y
dificultad para rastrear quién modificó qué. Como señalan \gs{}, el correo
electrónico no es un sistema de gestión de tareas (\principio{Gestión}).

\subsection{Objetivos}
\label{sec:objetivos}

Este artículo presenta un flujo de trabajo colaborativo que combina:
\begin{itemize}
    \item \herramienta{Git} para control de versiones
    \item \herramienta{GitHub} para almacenamiento y revisión
    \item \herramienta{Overleaf} para compilación en la nube
    \item \herramienta{VS Code} para edición local eficiente
\end{itemize}

El flujo integra los principios de buenas prácticas de \gs{},
incluyendo \principio{Automatización}, \principio{Directorios},
\principio{Abstracción} y \principio{Documentación}.
