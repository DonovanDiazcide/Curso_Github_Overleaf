\section{Resultados}
\label{sec:resultados}

Implementamos el flujo de trabajo propuesto durante el desarrollo de este
artículo. A continuación presentamos los resultados observados.

\subsection{Beneficios de la colaboración con Git}
\label{sec:beneficios-git}

Durante las dos horas del taller, el equipo logró:
\begin{itemize}
    \item Configurar un entorno de trabajo compartido
    \item Realizar múltiples contribuciones en paralelo usando ramas
    \item Resolver conflictos de manera sistemática
    \item Mantener un historial completo de cambios
\end{itemize}

\subsection{Mapeo de principios G\&S al flujo de trabajo}
\label{sec:mapeo-gs}

La Tabla~\ref{tab:mapeo-gs} muestra cómo cada principio de \gs{} se
implementa en el flujo de trabajo propuesto.

\begin{table}[h]
\centering
\caption{Mapeo de principios de \gs{} al flujo de trabajo}
\label{tab:mapeo-gs}
\begin{tabular}{@{}lll@{}}
\toprule
\textbf{Principio G\&S} & \textbf{Implementación} & \textbf{Herramienta} \\
\midrule
Automatización (Cap.~2) & \comando{make} compila todo & Makefile \\
Control de versiones (Cap.~3) & \comando{make check} antes de push & Git \\
Directorios (Cap.~4) & Separación entrada/salida & .gitignore \\
Claves (Cap.~5) & Convención \comando{sec:}, \comando{fig:}, \comando{tab:} & \LaTeX{} \\
Abstracción (Cap.~6) & Comandos reutilizables & \archivo{main.tex} \\
Documentación (Cap.~7) & Código auto-documentado & Todos \\
Gestión (Cap.~8) & GitHub Issues + commits & GitHub \\
\bottomrule
\end{tabular}
\end{table}

\subsection{Comparación con métodos tradicionales}
\label{sec:comparacion}

La Tabla~\ref{tab:comparacion} compara el flujo tradicional con el propuesto.

\begin{table}[h]
\centering
\caption{Comparación de flujos de trabajo colaborativo}
\label{tab:comparacion}
\begin{tabular}{@{}lcc@{}}
\toprule
\textbf{Característica} & \textbf{Tradicional} & \textbf{Git + Overleaf} \\
\midrule
Control de versiones & Manual & Automático \\
Historial de cambios & Limitado & Completo \\
Trabajo simultáneo & Difícil & Fácil \\
Resolución de conflictos & Ad-hoc & Sistemática \\
Backup & Manual & Automático \\
Revisión de cambios & Por correo & Pull Requests \\
Compilación & Manual (4 pasos) & \comando{make} (1 paso) \\
\bottomrule
\end{tabular}
\end{table}
