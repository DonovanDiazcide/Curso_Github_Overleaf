\section{Conclusión}

En este artículo presentamos un flujo de trabajo colaborativo para la 
escritura de artículos académicos en LaTeX, combinando Overleaf, GitHub 
y VS Code.

\subsection{Contribuciones principales}

Las principales contribuciones de este trabajo son:
\begin{enumerate}
    \item Un flujo de trabajo que aprovecha las fortalezas de cada herramienta
    \item Guías paso a paso para la configuración del entorno
    \item Estrategias para resolver conflictos de manera sistemática
    \item Uso de ramas para mantener versiones alternativas del documento
\end{enumerate}

\subsection{Limitaciones}

El flujo propuesto requiere que al menos un miembro del equipo tenga 
cuenta Premium de Overleaf para la sincronización con GitHub. Además, 
existe una curva de aprendizaje inicial para usuarios no familiarizados 
con Git.

\subsection{Trabajo futuro}

Como trabajo futuro, se podría explorar:
\begin{itemize}
    \item Integración con sistemas de gestión de referencias como Zotero
    \item Automatización de compilación con GitHub Actions
    \item Plantillas pre-configuradas para diferentes journals
\end{itemize}

\subsection*{Agradecimientos}

Agradecemos a todos los participantes del taller por su entusiasmo 
y colaboración durante el desarrollo de este ejercicio práctico.
