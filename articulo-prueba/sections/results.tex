\section{Resultados}

Implementamos el flujo de trabajo propuesto durante el desarrollo de este 
artículo. A continuación presentamos los resultados observados.

\subsection{Beneficios de la colaboración con Git}

Durante las dos horas del taller, el equipo logró:
\begin{itemize}
    \item Configurar un entorno de trabajo compartido
    \item Realizar múltiples contribuciones en paralelo
    \item Resolver conflictos de manera sistemática
    \item Mantener un historial completo de cambios
\end{itemize}

\subsection{Comparación con métodos tradicionales}

La Tabla \ref{tab:comparacion} muestra las diferencias entre el flujo 
tradicional (envío de archivos por correo) y el flujo propuesto.

\begin{table}[h]
\centering
\caption{Comparación de flujos de trabajo colaborativo}
\label{tab:comparacion}
\begin{tabular}{|l|c|c|}
\hline
\textbf{Característica} & \textbf{Tradicional} & \textbf{Git + Overleaf} \\
\hline
Control de versiones & Manual & Automático \\
\hline
Historial de cambios & Limitado & Completo \\
\hline
Trabajo simultáneo & Difícil & Fácil \\
\hline
Resolución de conflictos & Ad-hoc & Sistemática \\
\hline
Backup & Manual & Automático \\
\hline
Revisión de cambios & Por correo & Pull Requests \\
\hline
\end{tabular}
\end{table}

\subsection{Observaciones}

El principal desafío fue la curva de aprendizaje inicial de Git. 
Sin embargo, una vez dominados los comandos básicos, el flujo de trabajo 
resultó más eficiente que los métodos tradicionales.
