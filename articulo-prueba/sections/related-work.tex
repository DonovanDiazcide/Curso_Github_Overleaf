\section{Trabajos Relacionados}

Existen diversas aproximaciones a la colaboración en documentos académicos. 
A continuación revisamos las más relevantes.

\subsection{Herramientas de edición colaborativa}

Google Docs y Microsoft Word Online permiten edición simultánea, pero carecen 
de soporte nativo para LaTeX. Overleaf \citep{overleaf2024docs} resuelve esto 
parcialmente al ofrecer un editor LaTeX en línea con colaboración en tiempo real.

\subsection{Control de versiones en academia}

El uso de Git en investigación ha crecido significativamente. 
\citet{perez2024github} demuestran que GitHub facilita la reproducibilidad 
y colaboración en laboratorios de investigación.

\subsection{Flujos de trabajo híbridos}

Algunos equipos combinan múltiples herramientas. El enfoque que presentamos 
en este artículo sigue esta línea, integrando Overleaf, GitHub y editores locales.
