\section{Metodología}

En esta sección describimos las herramientas utilizadas y el flujo de trabajo 
propuesto para la colaboración en artículos académicos con LaTeX.

\subsection{Herramientas utilizadas}

El flujo de trabajo propuesto se basa en cuatro herramientas principales:

\begin{itemize}
    \item \textbf{Git}: Sistema de control de versiones distribuido que permite 
    rastrear cambios en archivos de texto, crear ramas para desarrollo paralelo, 
    y fusionar contribuciones de múltiples autores.
    
    \item \textbf{GitHub}: Plataforma en la nube que aloja repositorios Git, 
    proporcionando respaldo automático, Pull Requests para revisión de cambios, 
    e Issues para gestión de tareas.
    
    \item \textbf{VS Code}: Editor de código con extensión LaTeX Workshop que 
    permite compilación local automática y vista previa del PDF en tiempo real.
    
    \item \textbf{Overleaf}: Editor LaTeX en línea con compilador en la nube, 
    utilizado como capa de verificación final mediante sincronización con GitHub.
\end{itemize}

\subsection{Flujo de trabajo}

El ciclo diario de trabajo sigue el orden: \textbf{Local $\rightarrow$ GitHub 
$\rightarrow$ Overleaf}.

\begin{enumerate}
    \item \textbf{Obtener cambios}: \texttt{git pull origin main}
    \item \textbf{Editar localmente}: Escribir y compilar en VS Code
    \item \textbf{Preparar cambios}: \texttt{git add archivo.tex}
    \item \textbf{Guardar snapshot}: \texttt{git commit -m "descripción"}
    \item \textbf{Subir a GitHub}: \texttt{git push origin main}
    \item \textbf{Verificación final}: Sincronizar Overleaf con GitHub
\end{enumerate}
